\chapter*{Úvod}

Webové technologie procházejí velkými evolučními proměnami. Internet
už není ono nové a neprobádané místo jako v 90. letech 20.
století, ale naopak se stal běžnou a často i nepostradatelnou
součástí života obyčejných lidí. Technologie se posunují vpřed
jednotně na základě konsensu lídrů trhu, aplikace se přesouvají z
operačního systému do prohlížeče a vyvíjí se pro ně uživatelská
rozhraní s ohledem na běžné lidi. Ti se vlastně dostali do středu
veškerého zájmu. Jestliže dříve byl internet místem spíše pro
technicky založené, nyní se snaží oslovit širokou veřejnost. Lidem se
nabízí místa, kde mohou sdílet fotografie s přáteli a kde zjistí, co
jejich známí zrovna dělají, poslouchají, tvoří, čtou nebo sledují. Internet žije
okamžitou přítomností a zaznamenáváním této přítomnosti.

Různé zajímavé služby otevřeně poskytují svá data a na těch se staví
aplikace s dříve nepředstavitelnými komplexními funkcemi. Nejlépe to lze vidět
asi na mapách, které prošly neskutečným vývojem a dnes díky nim (když
zmíním jen zlomek jejich možností) například pohodlně naplánujete
cestu autem, zjistíte počasí v dané lokalitě, vyhledáte nejbližší
pekárnu, nebo si prohlédnete fotografie jiných lidí z míst, kam na
léto plánujete dovolenou. Na základě takových online map vznikají
mnohé specializované služby -- z těch nejzajímavějších v českém
prostředí bych zmínil například server bezrealitky.cz, jenž na
mapových podkladech zobrazuje realitní objekty. Nechal jsem se
inspirovat a na základě map jsem i já postavil systém, který využívá
mnoha otevřeně nabízených služeb a integruje je do komplexní aplikace
s užitkem pro běžného uživatele.

Evidencí geografických tras uživatele je tedy myšlena webová aplikace,
jež umožňuje uživateli interaktivně zaznačit do mapových podkladů trasu,
kterou plánuje absolvovat či kterou již absolvoval například na kole,
pěšky, na bruslích nebo během. Nic samozřejmě uživateli nebrání
evidovat i jiné, delší trasy, systém by měl být však primárně určen
pro výše naznačené lokální sportovní využití.

Bakalářská práce se zabývá analýzou požadavků na aplikaci, výběrem
technologií, mapových podkladů, \ldots

DOPSAT STRUCNE PRULET OBSAHEM BP (KRATKY ODSTAVECEK)

(TOTO MOZNA, UZ I TAK JE UVOD DOST DLOUHY) DOPSAT KRATCE A STRUCNE
ODKAZY NA POUZITOU LITERATURU A OBJASNENI PROC JSEM JI POUZIL (DVE VETY)

\chapter{Proměna internetu}

Dvacáté století nám připravilo univerzální platformu s obrovským
potenciálem. Lidé spojili inovativní myšlenky z mnoha směrů informatiky,
vybudovali fungující síť, úspěšně ji rozšířili po celé planetě
a dokázali z ní vytvořit každodenního pomocníka, bez kterého si život
už příliš představit nedokážeme. Kolem roku 2000 bylo na světě kolem
250~000~uživatelů internetu a toto číslo dále stoupalo exponenciálně.

Do nového století jsme však stále vstupovali spíše s pocitem, že
internet je něco nového a že tušíme minimum o tom, jaké jsou jeho
možnosti. Webové technologie byly celkem nesmělé, statické a pasivní.
Trh s prohlížeči, hlavními katalyzátory vývoje, byl nestabilizovaný.
Namísto toho, aby velcí hráči internetového trhu spolupracovali a v
navrhovaných novinkách se pokoušeli o konsensus, byly technologie
proprietárně uzamykány.

Až dnes se postupně dostáváme do doby, kdy můžeme pracovat s
relativně jednodtně podporovanými technologiemi, kdy se prosazují
spíše otevřená řešení a kdy začínají internetové subjekty
spolupracovat mezi sebou a více využívat možností propojení pomocí
internetu. Evoluce přinesla novou éru webu, brzy pojmenovanou jako
{\it Web~2.0}. Začaly se objevovat složité interaktivní aplikace přímo
ve webovém prohlížeči, stránky přestaly být izolovanými ostrůvky
statických informací, ale naopak mezi sebou začaly kooperovat,
poskytovat si navzájem data a funkcionalitu. Masivní rozšíření
internetu mezi běžné uživatele podnítilo vznik tzv. {\it sociálních
služeb}, kde mohou lidé sdílet informace mezi sebou a snadno
komunikovat, a také přineslo požadavek na propracovanější a
příjemnější uživatelské rozhraní webových aplikací.

Do kontextu tohoto nového proudu jsem se rozhodl zasadit svou práci.
Má aplikace využívá mnoha služeb a datových zdrojů
poskytovaných otevřeně a online, spojuje je dohromady a staví na nich
novou službu pro běžného netechnického uživatele. Taková architektura
se v terminologii Web~2.0 označuje běžně jako {\it mashup}
(míchanice). Navíc nabízí interaktivní uživatelské rozhraní a snaží
se mu poskytnout prostor pro sdílení svých dat s ostatními.

\chapter{Specifikace a analýza požadavků}

Jak jsem již pro osvětlení zmínil v úvodu práce, evidencí
geografických tras uživatele se rozumí webová aplikace, umožňující
člověku pohodlně a interaktivně zaznačit do mapových podkladů trasu,
kterou plánuje absolvovat či kterou již absolvoval například na kole,
pěšky, na bruslích nebo během. Nic samozřejmě uživateli nebrání
evidovat i jiné, delší trasy, systém by měl být však primárně určen
pro výše naznačené lokální sportovní využití.

Aplikace by měla být službou, kam se uživatel zaregistruje a potom,
přihlášen na svůj účet, může zakládat trasy svých výletů. Ty má
možnost ukládat a zpětně prohlížet. Trasy jako takové neposkytují jen
interaktivní záznam cesty na mapovém podkladu, ale také statistiky a
běžné informace o trase. Zde je prostor pro kombinaci s dalšími
podklady –- například s informacemi o nadmořské výšce terénu lze
uživateli poskytnout navíc výškový profil jeho trasy. Uživatel má
možnost trasy i plánovat. V tomto režimu aplikace vykazuje statistiky
trasy a může využít zase jiných datových zdrojů, aby poskytnula lepší
obraz o tom, co může například běžce na trase potkat (to může být
opět výškový profil, ale také např. předpověď počasí pro místo trasy
nebo vrstva s fotografiemi místních zajímavostí či panoramat).

Integrace s jinými zdroji dat potom může zasahovat i do zcela jiných
sfér –- např. by mohlo být možné implementovat výměnu dat s přenosným
GPS zařízením.

\section{Cílení projektu}
svet vs čr

Území Česka a Slovenska má jednu z nejdokonalejších a nejhustších sítí turistického 
značení pro pěší turistiku. Podobné značení má také Polsko, ale jinak je takováto síť prakticky 
světově unikátní. To znamená pro mou aplikaci především skutečnost, že pokud chce poskytovat 
možnost zobrazení těchto tras, musí využít lokálního poskytovatele mapových podkladů. Na 
druhou stranu bude potom funkčnost systému omezena prakticky na jeden stát, protože kvalitní 
a podrobné mapové podklady místních poskytovatelů nejsou v zásadě celosvětové.

\chapter{Výběr aplikačních rozhraní a zdrojů dat}
\ldots

\section{Specifika práce na základě API}
\subsection{Výhody a úskalí závislosti na externích API}
vyhody a nevyhody
\subsection{Formy API}
javascript, rest, obrázek, xml, json, \ldots
technologie nerozepisovat podrobne, jen se o nich zminit a vysvetlit
fungovani samotneho spojeni

\section{Mapové podklady}
V oblasti mapových technologií na internetu proběhl v
posledních letech opravdový zlom. Vše začalo 8. 2.
2005\footnote{http://googleblog.blogspot.com/2005/02/mapping-your-way.html},
kdy Google spustil své revolučně zpracované, interaktivní Google Maps.
V řetězové reakci si postupně i další provozovatelé online map začali
uvědomovat skrytý potenciál této služby (prodej regionální reklamy,
cílená reklama, partnerství s jízdními řády apod.) a začali také
investovat velké částky do její modernizace. Na českém internetu
navíc vznikly časem tři velké a velmi kvalitní mapové servery, což je
ve světě celkem jedinečný úkaz a působí ještě unikátněji,
přihlédneme-li k velikosti a významu naší země.

Ještě v roce jejich vydání představil Google jako první u svých
map aplikační rozhraní pro použití mapových podkladů i na jiných
webech. Vydání API se setkalo s obrovským ohlasem a nadobro změnilo
web, jak jsme ho znali. Po celém internetu se začaly objevovat
interaktivní mapy -- od jednoduchých orientačních výřezů po aplikace
na mapách kompletně založené. Konkurence ani nyní nespala a odpověděla
svými vlastními API.

V následujících odstavcích se pokusím popsat specifika, výhody a
nevýhody jednotlivých mapových podkladů, jež jsem bral pro svou práci
v úvahu. Je nutné podotknout, že samotných mapových serverů je mnohem
více (např. Mapy iDNES.cz, Ask.com Maps \& Directions, Multimap,
NAVTEQ Map24, Bing Maps, aj.), ale nelze jejich služeb využít,
protože API vůbec nenabízejí, nebo má velmi omezené možnosti. Také
jsem vynechal možnost získat geografická data jejich zakoupením přímo
od dodavatelů.

\subsection{Google Maps společnosti Google}
Průkopník v oblasti online map, Google, nabízí samozřejmě mapový
server i API již hodně dlouho, takže jeho služby jsou v mnoha směrech
nejvyzrálejší. API je pod neustálým vývojem a v době psaní práce
Google pracuje na jeho třetí
verzi\footnote{http://code.google.com/intl/cs/apis/maps/documentation/v3/}.

Nelze se však nechat unést jeho možnostmi a je nutné zaměřit se i na
jiné rysy, důležité pro tuto práci. Mezi takové patří
například skutečnost, že do češtiny začala být služba lokalizována až
nedávno\footnote{http://www.lupa.cz/clanky/mapy-google-v-cestine-realita-nebo-zbozne-prani/}.
Dnes je již sice míra integrace map do českého prostředí na velmi
dobré úrovni, ale z hlediska mapových podkladů má jednu velkou mezeru
-- turistická data. Google poskytuje mapy globálně a proto se mu v
nich velmi špatně odráží specifika jednotlivých zemí. K dispozici
jsou sice terénní mapy s vrstevnicemi, ale neexistuje možnost
zobrazit na nich české turistické trasy a cyklostezky.

Jinak jsou podklady kvalitní, i když někdy méně přesné, než u
lokálních mapových služeb. Google používá kombinaci několika zdrojů
geografických map, přičemž většinu českých podkladů získává od
dodavatele GEODIS Brno. Předností map je samozřejmě dostupnost
podkladů pro celý svět a lákavá je rovněž představa možného budoucího 
napojení aplikace např. v podobě vrstvy na program Google
Earth\footnote{Google Earth je multiplatformní program společnosti
Google představující virtuální online glóbus. Nabízí pohled na zemi
jako z družice, virtuální 3D modely některých měst, detailní snímky
zajímavých míst po celém světě a umožňuje překryv mapových podkladů
tzv. vrstvami poskytujícími další informace. Google jej nabízí v
několika variantách, z nichž základní je zdarma.}.

\subsection{Mapy.cz společnosti Seznam.cz}
Mapy.cz byly prvním ryze českým projektem v oblasti nových online map
a dodnes jsou lídrem lokálního trhu. Stejně jako za Google Maps stojí
i za těmito mapami silná společnost. Budoucnost serveru a případný
další vývoj API je celkem jistý. Seznam.cz se na rozdíl od všech
ostatních českých portálů profiloval po vzoru Google spíše do
společnosti, jež svou budoucnost spojuje s technologickým pokrokem,
než do mediálního vydavatelství jako například Centrum Holdings.
Odhadnutelné záměry potvrdili uveřejněním zprávy o vývoji nového
API\footnote{http://mapy.cz.sblog.cz/2009/02/18/29} v čase tvorby
této práce.

Současný stav API je ale celkem nešťastný. Aplikační rozhraní nabízí
jen omezenou škálu funkcí, omezené mapové podklady oproti službě
Mapy.cz a samotná práce s funkcemi API působí na vývojáře poněkud
těžkopádně. Jeho licenční podmínky navíc nejsou tak volné jako u
ostatních API a požadují registraci klíče nikoliv na doménu, ale
přímo na unikátní URL, kde se má mapa nacházet. To jej pro tvorbu
složitější aplikace prakticky vyřazuje ze hry. V podmínkách je také
omezení na 1000 zobrazení denně a zákaz provozu map pro komerční
užití, což v ranné fázi projektu není velkou překážkou, ale pro
budoucí rozvoj projektu ano\footnote{http://api.mapy.cz/keygen}.

Nové API čtvrté verze vyvíjí v Seznam.cz Ondřej Žára, autor známého
nástroje pro tvorbu databázových schémat, {\it WWW SQL
Designer}\footnote{http://code.google.com/p/wwwsqldesigner/}. Bohužel
rozhraní je zatím stále dost nestabilní a podle jeho slov ani jemu
stále ještě nejsou známy nové licenční podmínky.

Mapy.cz jsou připraveny kombinací geografických dat od PLANstudio a
GEODIS Brno. Turistická mapa je nejkvalitnější podobnou mapou na českém
internetu. Seznam.cz ji poskytuje na základě dat společnosti SHOCart,
známou svými papírovými turistickými a cyklistickými publikacemi. Je
škoda, že nepoužitelné API v tomto případě brání využít tak kvalitní
podklady.

\subsection{Amapy.cz společnosti Centrum Holdings}
Amapy.cz se na svět dostaly v roce 2006 pod hlavičkou portálu
Atlas.cz\footnote{http://management.blog.lupa.cz/2006/11/05/spetka-koreni-ze-zakulisi-projektu-novych-atlasich-map/}.
Ihned od představení bylo jasné, že se s nimi musí na českém
trhu počítat -- zpracování bylo profesionální a spolu s mapami přišlo
i první na funkce bohaté, dobře dokumentované české mapové API. Vývoj
však postupně ustával a po tom, co byl Atlas.cz s sjednocen s
Centrum.cz pod hlavičku Centrum Holdings, již nelze kolem API
pozorovat vůbec žádnou činnost ze strany provozovatele. Celou službu
původně zpracoval Daniel Steigerwald, který tyto informace pro mou
práci potvrdil.

API je však opravdu velmi dobře použitelné a mapové podklady
kvalitní, připravené ve spolupráci s firmou DPA. I přes API lze
dokonce zobrazovat vrstvy s turistickými a cyklistickými značkami a už i
zcela základní mapa disponuje vrstevnicemi. Aplikační rozhraní nabízí
funkce, jež nelze najít ani u Google Maps API a podporuje několik
souřadnicových systémů naráz, což je výhodné při spolupráci s jinými
službami (každá požaduje body v jiném formátu).

Specifikem API je integrovaný JavaScriptový framework MooTools 1.11.
Výhodou je, že po vložení API do stránky lze přímo využít všech výhod
frameworku a není nutné nějaký připojovat dodatečně. Nevýhodou je
nemožnost vlastního výběru frameworku a také ustrnutí vývoje API,
protože v důsledku toho nebyl průběžně framework obnovován a zůstal
ve verzi 1.11, ačkoliv během psaní práce byl k dispozici již ve velmi
odlišné verzi 1.2.3.

\subsection{Otevřený projekt OpenStreetMap}
OpenStreetMap je otevřený projekt, který se snaží vytvořit volně
dostupná geografická data. Získává je integrací dat z různých zdrojů
a především individuálním sběrem dat pomocí GPS zařízení. Mnoho
institucí, ogranizací a dokonce i firem uvolnilo svá data pod licencí
kompatibilní s OpenStreetMap, aby tomuto projektu pomohli.

Kvalita mapových podkladů pro ČR však není zrovna nejlepší a
pro účely aplikace se nehodí ani forma jejich zobrazení. Mapy sice
obsahují například polohu sloupů elektrického vedení nebo přesné
hranice lesů, vrstevnice nebo turistické značky a cyklostezky však
nepodporují.

OpenStreetMap je zajímavý počin a do budoucna možná perspektivní, ale
o jeho použití ve své práci jsem příliš neuvažoval. Uvedl jsem jej
pro úplnost jako alternativní a otevřený zdroj geografických dat,
jenž by v budoucnosti mohl nabýt na relevanci.

\subsection{Výběr API s ohledem na požadavky a cílení práce}
Z charakteru aplikace, jež je předmětem této práce, celkem jasně
vyplývá, že výběr mapových podkladů je jedním z nejdůležitějších
rozhodnutí, které bude mít vliv na celý další vývoj projektu.
Soustředil jsem se proto opravdu pečlivě na charakteristiky
jednotlivých API a dlouze zvažoval.

Vzhledem k faktům zmíněným v předcházejících odstavcích jsem
postupně nabyl dojmu, že ani jedno řešení rozhodně nelze favorizovat a
spíše bude nutné vybrat nejméně bolestivý kompromis. Vybíral jsem
podle třech hlavních kritérií:
\begin{itemize}
	\item Možnosti a funkce API,
	\item kvalita mapových podkladů s důrazem na turistické
	mapy a možnosti zobrazovat české turistické značky a cyklictické
	trasy,
	\item zázemí poskytovatele a budoucnost API.
\end{itemize}

Google nabízí bezkonkurenční API a dokonce v současné době
představuje jeho zcela novou verzi, ale jeho mapové podklady jsou pro turistiku v
českých podmínkách naprosto nedostatečné. Mapy.cz naopak disponují
vynikajícími mapovými podklady, jenomže jejich aplikační rozhraní je
velmi chabé z hlediska funkcionality a ještě mnohem více omezující v
oblasti licenčních podmínek jeho použití. Seznam.cz sice vyvíjí nové
API, ale to je zatím velmi nestabilní a jeho licence je stále
neznámá. Pro svou práci jsem nakonec vybral podklady od Centrum
Holdings, protože poskytují dobré turistické mapy a mají bohatou
škálu funkcí. S faktem, že vývoj aplikačního rozhraní je již několik
let zcela mrtvý a jeho budoucnost je nejistá, mi nezbylo než se
smířit a počítat s tím, že v budoucnu ji možná bude zapotřebí přepsat
pro jiné API. V tomto ohledu bych vzhlížel k vývoji nového API v Seznam.cz.




\section{Výškový profil}
\subsection{Nadmořská výška}
geonames a vyskopis
\subsection{Grafy}
gugl a dalsi knihovny

\section{Obraz z terénu}
\subsection{Fotografie}
panoramio atd.
\subsection{Webkamery}
\ldots

\section{Počasí}
\ldots

\section{Užitečná turistická data}
\subsection{Wikipedia}
\ldots
\subsection{Zdroje POI}
\ldots
\subsection{MHD}
\ldots
\subsection{Občerstvení}
\ldots

\chapter{Interoperabilita aplikace}
\ldots

\section{Autentizace uživatele}
openid

\section{Import a export dat}
gpx, kml
\subsection{GPS eXchange Format}
gpx
\subsection{Keyhole Markup Language}
kml

\section{Mikroformáty}
\ldots

\chapter{Použité technologie}
celkem klasicke kvuli prenosnosti a urovni hostingu +
udrzovatelnosti do budoucna a podpory nastroju, api apod.\ldots

\section{Uživatelské rozhraní webové stránky}
šablona, \ldots
\subsection{Sémantický dokument s HTML}
zaklady + se malinko odkazat i na mikroformaty

\subsection{Volba mezi HTML a XHTML}
webylon\ldots

\subsection{Styly pomocí CSS}
\ldots

\subsection{Jazyk ECMAScript}
javascript\ldots

\subsection{Framework MooTools}
i jine fw, vylepseni, odstineni, dom, vyhody fw obecne

\section{Serverová část}
generovani ux na serveru
\subsection{Jazyk PHP}
\ldots
\subsection{Systém řízení báze dat MySQL}
\ldots
\subsection{Nette Framework}
nette, dibi\ldots

\section{AJAX}
vysvetleni

\section{Práce s API}
pouzivani api klicu atd.\ldots
\subsection{Univerzální jazyk XML}
zakladni uvedeni do problematiky, vyhody nevyhody\ldots
\subsection{Lidsky čitelný JSON}
vyhody nevyhody\ldots

\chapter{Řešení systému}
popis celkoveho reseni systemu, detaily v kapitolkach o jednotlivych
problemech jako pocitani ruznych bodu na mape, reseni prodlev dat
dodavanych z api, zdvojovani api apod.

\chapter{Implementace}
navrh databaze, mozna use case, ale do toho se mi moc nechce,
rozepsani detailneji nez v minule kapitole o tom kde je javascript,
co dela, jak je implementovano ukladani, kresleni car na tu mapu
apod. detaily

\chapter{Vývoj a testování}
verze vyvojovych nastroju, verze mootools, nette, eclipse, apache,
php, mysql, testovane prohlizece

\chapter*{Závěr}
dosazene vysledky a prinos prace

Závěrečná kapitola obsahuje zhodnocení dosažených výsledků se zvlášť
vyznačeným vlastním přínosem studenta. Povinně se zde objeví i
zhodnocení z pohledu dalšího vývoje projektu, student uvede náměty
vycházející ze zkušeností s řešeným projektem a uvede rovněž
návaznosti na právě dokončené projekty (řešené v rámci ostatních
bakalářských prací v daném roce nebo na projekty řešené na externích
pracovištích).
