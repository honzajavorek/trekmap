\chapter*{Úvod}

Webové technologie procházejí velkými evolučními proměnami. Internet
už není ono nové a neprobádané místo jako v 90. letech 20.
století, ale naopak se stal běžnou a často i nepostradatelnou
součástí života obyčejných lidí. Technologie se posunují vpřed
jednotně na základě konsensu lídrů trhu, aplikace se přesouvají z
operačního systému do prohlížeče a vyvíjí se pro ně uživatelská
rozhraní s ohledem na běžné lidi. Ti se vlastně dostali do středu
veškerého zájmu. Jestliže dříve byl internet místem spíše pro
technicky založené, nyní se snaží oslovit širokou veřejnost. Lidem se
nabízí místa, kde mohou sdílet fotografie s přáteli a kde zjistí, co
jejich známí zrovna dělají, poslouchají, tvoří, čtou nebo sledují. Internet žije
okamžitou přítomností a zaznamenáváním této přítomnosti.

Různé zajímavé služby otevřeně poskytují svá data a na těch se staví
aplikace s dříve nepředstavitelnými komplexními funkcemi. Nejlépe to lze vidět
asi na mapách, které prošly neskutečným vývojem a dnes díky nim (když
zmíním jen zlomek jejich možností) například pohodlně naplánujete
cestu autem, zjistíte počasí v dané lokalitě, vyhledáte nejbližší
pekárnu, nebo si prohlédnete fotografie jiných lidí z míst, kam na
léto plánujete dovolenou. Na základě takových online map vznikají
mnohé specializované služby -- z těch nejzajímavějších v českém
prostředí bych zmínil například server www.bezrealitky.cz, jenž na
mapových podkladech zobrazuje realitní objekty. Nechal jsem se
inspirovat a na základě map jsem i já postavil systém, který využívá
mnoha otevřeně nabízených služeb a integruje je do komplexní aplikace
s užitkem pro běžného uživatele.

Evidencí geografických tras uživatele je tedy myšlena webová aplikace,
jež umožňuje uživateli interaktivně zaznačit do mapových podkladů trasu,
kterou plánuje absolvovat či kterou již absolvoval například na kole,
pěšky, na bruslích nebo během. Nic samozřejmě uživateli nebrání
evidovat i jiné, delší trasy, systém by měl být však primárně určen
pro výše naznačené lokální sportovní využití.

Bakalářská práce se zabývá analýzou požadavků na aplikaci, výběrem
technologií, mapových podkladů, \ldots

DOPSAT STRUCNE PRULET OBSAHEM BP (KRATKY ODSTAVECEK)

(TOTO MOZNA, UZ I TAK JE UVOD DOST DLOUHY) DOPSAT KRATCE A STRUCNE
ODKAZY NA POUZITOU LITERATURU A OBJASNENI PROC JSEM JI POUZIL (DVE VETY)

\chapter{Proměna internetu}

Dvacáté století nám připravilo univerzální platformu s obrovským
potenciálem. Lidé spojili inovativní myšlenky z mnoha směrů informatiky,
vybudovali fungující síť, úspěšně ji rozšířili po celé planetě
a dokázali z ní vytvořit každodenního pomocníka, bez kterého si život
už příliš představit nedokážeme. Kolem roku 2000 bylo na světě kolem
250~000~uživatelů internetu a toto číslo dále stoupalo exponenciálně.

Do nového století jsme však stále vstupovali spíše s pocitem, že
internet je něco nového a že tušíme minimum o tom, jaké jsou jeho
možnosti. Webové technologie byly celkem nesmělé, statické a pasivní.
Trh s prohlížeči, hlavními katalyzátory vývoje, byl nestabilizovaný.
Namísto toho, aby velcí hráči internetového trhu spolupracovali a v
navrhovaných novinkách se pokoušeli o konsensus, byly technologie
proprietárně uzamykány.

Až dnes se postupně dostáváme do doby, kdy můžeme pracovat s
relativně jednodtně podporovanými technologiemi, kdy se prosazují
spíše otevřená řešení a kdy začínají internetové subjekty
spolupracovat mezi sebou a více využívat možností propojení pomocí
internetu. Evoluce přinesla novou éru webu, brzy pojmenovanou jako
{\it Web~2.0}. Začaly se objevovat složité interaktivní aplikace přímo
ve webovém prohlížeči, stránky přestaly být izolovanými ostrůvky
statických informací, ale naopak mezi sebou začaly kooperovat,
poskytovat si navzájem data a funkcionalitu. Masivní rozšíření
internetu mezi běžné uživatele podnítilo vznik tzv. {\it sociálních
služeb}, kde mohou lidé sdílet informace mezi sebou a snadno
komunikovat, a také přineslo požadavek na propracovanější a
příjemnější uživatelské rozhraní webových aplikací.

Do kontextu tohoto nového proudu jsem se rozhodl zasadit svou práci.
Má aplikace využívá mnoha služeb a datových zdrojů
poskytovaných otevřeně a online, spojuje je dohromady a staví na nich
novou službu pro běžného netechnického uživatele. Taková architektura
se v terminologii Web~2.0 označuje běžně jako {\it mashup}
(míchanice). Navíc nabízí interaktivní uživatelské rozhraní a snaží
se mu poskytnout prostor pro sdílení svých dat s ostatními.

\chapter{Specifikace a analýza požadavků}

Jak jsem již pro osvětlení zmínil v úvodu práce, evidencí
geografických tras uživatele se rozumí webová aplikace, umožňující
člověku pohodlně a interaktivně zaznačit do mapových podkladů trasu,
kterou plánuje absolvovat či kterou již absolvoval například na kole,
pěšky, na bruslích nebo během. Nic samozřejmě uživateli nebrání
evidovat i jiné, delší trasy, systém by měl být však primárně určen
pro výše naznačené lokální sportovní využití.

Aplikace by měla být službou, kam se uživatel zaregistruje a potom,
přihlášen na svůj účet, může zakládat trasy svých výletů. Ty má
možnost ukládat a zpětně prohlížet. Trasy jako takové neposkytují jen
interaktivní záznam cesty na mapovém podkladu, ale také statistiky a
běžné informace o trase. Zde je prostor pro kombinaci s dalšími
podklady –- například s informacemi o nadmořské výšce terénu lze
uživateli poskytnout navíc výškový profil jeho trasy. Uživatel má
možnost trasy i plánovat. V tomto režimu aplikace vykazuje statistiky
trasy a může využít zase jiných datových zdrojů, aby poskytnula lepší
obraz o tom, co může například běžce na trase potkat (to může být
opět výškový profil, ale také např. předpověď počasí pro místo trasy
nebo vrstva s fotografiemi místních zajímavostí či panoramat).

Integrace s jinými zdroji dat potom může zasahovat i do zcela jiných
sfér –- např. by mohlo být možné implementovat výměnu dat s přenosným
GPS zařízením.

\section{Cílení projektu}
svet vs čr

\chapter{Výběr aplikačních rozhraní a zdrojů dat}
\ldots

\section{Specifika práce na základě API}
\subsection{Úskalí a výhody závislosti na externích API}
vyhody a nevyhody
\subsection{Formy API}
javascript, rest, obrázek, xml, json, \ldots
technologie nerozepisovat podrobne, jen se o nich zminit a vysvetlit
fungovani samotneho spojeni

\section{Mapové podklady}
seznam, google, atlas, \ldots

\section{Výškový profil}
\subsection{Nadmořská výška}
geonames a vyskopis
\subsection{Grafy}
gugl a dalsi knihovny

\section{Obraz z terénu}
\subsection{Fotografie}
panoramio atd.
\subsection{Webkamery}
\ldots

\section{Počasí}
\ldots

\section{Užitečná turistická data}
\subsection{Wikipedia}
\ldots
\subsection{Zdroje POI}
\ldots
\subsection{MHD}
\ldots
\subsection{Občerstvení}
\ldots

\chapter{Interoperabilita aplikace}
\ldots

\section{Autentizace uživatele}
openid

\section{Import a export dat}
gpx, kml
\subsection{GPS eXchange Format}
gpx
\subsection{Keyhole Markup Language}
kml

\section{Mikroformáty}
\ldots

\chapter{Použité technologie}
celkem klasicke kvuli prenosnosti a urovni hostingu +
udrzovatelnosti do budoucna a podpory nastroju, api apod.\ldots

\section{Uživatelské rozhraní webové stránky}
šablona, \ldots
\subsection{Sémantický dokument s HTML}
zaklady + se malinko odkazat i na mikroformaty

\subsection{Volba mezi HTML a XHTML}
webylon\ldots

\subsection{Styly pomocí CSS}
\ldots

\subsection{Jazyk JavaScript}
ecmascript\ldots

\subsection{Framework MooTools}
i jine fw, vylepseni, odstineni, dom, vyhody fw obecne

\section{Serverová část}
generovani ux na serveru
\subsection{Jazyk PHP}
\ldots
\subsection{Systém řízení báze dat MySQL}
\ldots
\subsection{Nette Framework}
nette, dibi\ldots

\section{AJAX}
vysvetleni

\section{Práce s API}
pouzivani api klicu atd.\ldots
\subsection{Univerzální jazyk XML}
zakladni uvedeni do problematiky, vyhody nevyhody\ldots
\subsection{Lidsky čitelný JSON}
vyhody nevyhody\ldots

\chapter{Řešení systému}
popis celkoveho reseni systemu, detaily v kapitolkach o jednotlivych
problemech jako pocitani ruznych bodu na mape, reseni prodlev dat
dodavanych z api, zdvojovani api apod.

\chapter{Implementace}
navrh databaze, mozna use case, ale do toho se mi moc nechce,
rozepsani detailneji nez v minule kapitole o tom kde je javascript,
co dela, jak je implementovano ukladani, kresleni car na tu mapu
apod. detaily

\chapter{Vývoj a testování}
verze vyvojovych nastroju, verze mootools, nette, eclipse, apache,
php, mysql, testovane prohlizece

\chapter*{Závěr}
dosazene vysledky a prinos prace

Závěrečná kapitola obsahuje zhodnocení dosažených výsledků se zvlášť
vyznačeným vlastním přínosem studenta. Povinně se zde objeví i
zhodnocení z pohledu dalšího vývoje projektu, student uvede náměty
vycházející ze zkušeností s řešeným projektem a uvede rovněž
návaznosti na právě dokončené projekty (řešené v rámci ostatních
bakalářských prací v daném roce nebo na projekty řešené na externích
pracovištích).
